  \subsection{Principe de fonctionnement de notre solution}
      \paragraph{}
	  
% 	  \subsubsection{Clients}
% 	  \paragraph{}	  
% 	  \subsubsection{Vendeurs}
% 	  \paragraph{}	  
% 	  \subsubsection{Administrateurs}
	  \paragraph{}
	  	% Les factures d'un m\^eme compteur sont identifi\'ees gr\^ace au num\'ero de police.
	  	Les factures d'un utilisateur sont identifi\'ees gr\^ace \`a la r\'ef\'erence abonn\'ee fournie \`a l'inscription. Une fois connect\'e \`a l'application, le client dispose de la liste de ses factures. Il peut donc s\'electionner le ou les factures qu'il d\'esire solder et proc\'eder au paiement. Le progiciel G'd'Or g\'en\`ere tous les jours \`a une heure pr\'ecise un fichier contenant les factures impay\'ees de tous les compteurs de la SBEE. Ce fichier est reçu par le serveur d'application et ensuite charg\'e dans la  base de donn\'ee. Ainsi nous disposons des impay\'ees au niveau de tous les num\'eros de compteurs. A chaque fin de journ\'ee, \`a une heure pr\'ecise, l'application g\'en\`ere \`a son tour un fichier contenant les factures qui ont \'et\'e pay\'ees dans la journ\'ee et le total encaiss\'e. Nous consid\'erons ce ficher comme un bilan fait par notre application \`a Gd'OR. Ce fichier permet \`a Gd'Or de valider les paiements et d'apurer les comptes en fonction des informations fournies par les prestataires de solution de paiement. Tous les \'echanges de fichiers se feront via un canal s\'ecuris\'e (SFTP et VPN). Apres utilisations, ils sont archiv\'es de façon automatique afin de garder une trace des \'echanges et des flux de donn\'ees. Les erreurs notifi\'ees sont imm\'ediatement signal\'ees aux entit\'es concern\'ees et ils doivent y apporter des solutions le plus t\^ot possible. 
	  	
	  	Exemples de fichiers :
	  	\begin{itemize}
		  \item Impay\'ees
		  \item Pay\'es
		\end{itemize}

	  		  	
	  	L'architecture de la solution se pr\'esente comme suit :
		\begin{itemize}
		  \item les clients
		  \item internet
		  \item le systeme de paiement
		  \item les serveurs
		  \item les routeurs, parefeu
		\end{itemize}

  \subsection{Langages de développements et outils}
  
      \paragraph{}
	  Notre solution est une application web. Elle est donc accessible depuis tous les systèmes d'exploitations du moment qu'un navigateur et d'une connexion internet sont disponibles. Elle a \'et\'e r\'ealis\'e gr\^ace au framework Django.\\
	  
	  \textbf{Pourquoi utiliser un framework ?}\\
	  Lorsque l'on réalise un site Internet, on en revient toujours aux même étapes :
	  \begin{itemize}
	    \item réalisation et codage du design ;
	    \item réalisation des modules :
	    \begin{itemize}
	      \item réalisation du modèle de données concernant le module,
	      \item réalisation des formulaires d'ajout, modification et suppression des données :
	    \end{itemize}
	    \item réalisation des pages d'affichage du contenu du site ;
	    \item réalisation d'une administration pour gérer les modules ;
	    \item réalisation d'un espace utilisateur avec des droits sur l'accès aux données ;
	    \item mise en place de flux RSS/ATOM ;
	    \item mise en place d'un plan du site ;
	  \end{itemize}
	  \paragraph{}
	  Tout cela est relativement répétitif, et si, la première fois, ça peut paraître très amusant, on en arrive rapidement à faire des copier/coller, assez mauvaise méthode car source de nombreuses erreurs. Finalement on regroupe des morceaux de code en fonctions réutilisables. À ce moment, on se rapproche de plus en plus de la notion de framework ci-dessus. L'avantage d'utiliser un framework existant et surtout Open Source tel que Django, c'est que nous ne sommes pas les seuls à l'utiliser, et que les bugs sont donc corrigés plus rapidement, les améliorations sont exécutées par plusieurs personnes et de manière bien mieux réfléchie. C'est d'ailleurs tout l'intérêt d'utiliser un framework. En faire moins, pour en faire plus dans le même temps.\\
	  
	  
	  \textbf{Pourquoi Django ?}\\
	  Il existe de nombreux framework web, dans différents langages de programmation. Pourquoi utiliser spécifiquement Django et pas un autre ? Nombreuses sont les raisons qui motivent ce choix : 
	  \begin{itemize}
	    \item La simplicité d'apprentissage.
	    \item La qualité des applications réalisées.
	    \item La rapidité de développement.
	    \item La sécurité de l'application.
	    \item La facilité de maintenance des applications sur la durée.
	  \end{itemize}
	  Outres ces avantages on bénéficie de la clarté de Python, qui permet à plusieurs développeurs de travailler sur le même projet. Le style est imposé, donc tout le monde suit les mêmes règles, ce qui facilite les travaux en équipe et la clarté du code.
	  
	  \subsubsection{POSTGRESQL}
	    PostgreSQL est un système de gestion de base de données relationnelle et objet (SGBDRO). C'est un outil libre disponible selon les termes d'une licence de type BSD.
	    Ce système est concurrent d'autres systèmes de gestion de base de données, qu'ils soient libres (comme MariaDB et Firebird), ou propriétaires (comme Oracle, MySQL, Sybase, DB2, Informix et Microsoft SQL Server). Comme les projets libres Apache et Linux, PostgreSQL n'est pas contrôlé par une seule entreprise, mais est fondé sur une communauté mondiale de développeurs et d'entreprises.

	  \subsubsection{Ubuntu 16.04 LTS}
	    Ubuntu 16.04 LTS est la 6ème version LTS du système d'exploitation basé sur Linux, et comme nous en avons discuté ici, elle apporte plusieurs nouvelles fonctionnalités et améliorations. Compte tenu de leur cycle de support, les versions LTS conviennent mieux aux entreprises et aux utilisateurs finaux qui n'aiment pas mettre à jour leur système d'exploitation de temps en temps. Si vous êtes l'un d'entre eux, allez-y et prenez la publication.
	    

      \subsection{Environnement de Production}
	\paragraph{}
	  Debian est un système d'exploitation libre. Il est simple et est constitué de plus de 4000 paquets. Les paquets sont des composants logiciels précompilés conçus pour s'installer facilement sur la machine hôte. L'ouverture de Debian permet à plusieurs programmeurs de pouvoir identifier les failles de sécurité ou de créer plusieurs modules pour faire des tâches qui s'avèrent indispensables. C'est ainsi que la communauté de Debian devient de plus en plus robuste et flexible. Plusieurs systèmes d'exploitation sont dérivés de Debian dont Kali Linux spécialisé dans les tests d'intrusion. 
  
\section*{Conclusion}
		\paragraph{}
	  Dans ce chapitre, nous avons présenté les choix techniques opérés
	  ainsi que notre solution à travers sa modélisation, son principe de fonctionnement et les outils utilisés. 
	  Le chapitre suivant exposera les différents résultats et quelques critiques.	