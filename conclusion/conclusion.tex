\conclusion
\conclusion
		\paragraph{}
			La sécurité informatique passant par la sécurité de l'information, la sécurité de la vie privée n'est plus aujourd'hui un sujet inconnu. Aussi le hacking constitue-t-il aujourd'hui une arme de guerre assez dévastatrice et silencieuse. Dans cette vision, beaucoup d'outils de sécurité se développent de jour en jour afin de restreindre le champ d'attaque des pirates informatiques qui s'arment davantage. 
		\paragraph{}
			L'objet de ce mémoire a été de mettre en place un système de détection des attaques par webcam et de prévenir par des moyens simples mais efficaces les tentatives d'intrusion. Les solutions proposées proviennent des analyses éffectuées sur les systèmes piratés volontairement. Ainsi les utilisateurs de notre application peuvent être protégés des différentes menaces que nous couvrons.
		\paragraph{}
			Dans ce mémoire, nous avons tout d'abord fait une revue de littérature autour des intrusions informatiques. Nous avons ensuite réalisé la conception de la solution que nous avons proposée avant d'exposer les résultats
			et critiques de l'application que nous avons implémentée.
		\paragraph{}
			Bien que notre solution réponde au besoin énoncé, il faut noter certaines insuffisances telles que 
			l'inactivité de l'application dans les intervalles d'attente.  Bien que cela soit fait pour éviter l'utilisation abusive des ressources du système, il serait préférable de mettre le système en mode écoute. D'une part, un autre axe
			de recherche en ce qui concerne les travaux futurs sera d'explorer la possibilité de rendre 
			l'application accessible sur d'autres plateformes telles que Windows, MAC OS et toutes autres distributions Linux. D'autre part, on pourrait envisager de mettre la recherche de symptômes d'attaques en mode écoute en prenant des mesures autonomes. \cite{ehrig2006graph}