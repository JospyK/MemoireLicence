\conclusion
    \paragraph{}
	Une entreprise doit apprendre \`a fid\'eliser ses clients. On dit chez nous que le client est Roi. Pour une entreprise qui fournit des services \`a l'\'echelle nationale, savoir que chaque client qui vient en agence solder sa facture perd du temps, ce n'est pas une information agr\'eable. Ces pertes de temps peuvent \^etre expliqu\'ees par les longues queues, les probl\`emes de connexion.

    \paragraph{}
	L'objet de ce mémoire a donc été de proposer un prototype d'un syst\`eme de paiement \'electronique des factures d'\'electricit\'e de la SBEE dans un environnement s\'ecuris\'e. Il s'agit d'une part de l'application web r\'ealis\'ee et d'autre part de la proposition de l'environnement de la mise en production de cette application. La solution est accessible via internet et ne requiert aucun équipement ou installation spécifique chez l'utilisateur et plusieurs mesures ont \'et\'e prises pour assurer son fonctionnement en toute qui\'etude.

    \paragraph{}
	Dans ce mémoire, nous avons tout d'abord fait la revue de littérature autour du progiciel permettant le paiement des factures et quelques notions par rapport \`a la s\'ecurit\'e des applications Web. Nous avons ensuite réalisé la conception de la solution et propos\'e une architecture pour sa mise en production. Enfin nous avons expos\'e les résultats et critiques de l'application r\'ealis\'ee.
    \paragraph{}
	Bien que notre solution réponde au besoin énoncé, il faut noter certaines insuffisances telles que les doublures lors des paiements des factures. La disponibilit\'e limit\'ee de l'application pour r\'eduire les cas de doublures. D'une part, un autre axe de recherche en ce qui concerne les travaux futurs sera de permettre l'achat de cr\'edit pr\'epay\'e depuis la m\^eme plateforme.

    \paragraph{}
	Notre travail s'est d\'eroul\'e dans les locaux de la SBEE. Cependant nous tenons \`a pr\'eciser que la Soci\'et\'e Nationale des Eaux du B\'enin (SONEB) et la Soci\'et\'e B\'eninoise d'Energie Electrique ont exactement la m\^eme organisation et donc le m\^eme principe de fonctionnement. En r\'ealit\'e Gd'Or est aussi le logiciel m\'etier utilis\'e par la SONEB. Notre travail est donc tout aussi valide pour les factures d'eau  que les factures d'\'electricit\'e. Une entente entre ces deux soci\'et\'es permettrait d'avoir depuis la m\^eme plateforme les deux types de factures. Ce travail devrait donc \^etre pr\'esent\'e \`a la SONEB afin qu'ils puissent aussi en b\'en\'eficier.

    \paragraph{}
	Nous souhaiterions que les diff\'erentes r\`egles par rapport aux vendeurs soient d\'efinies afin que nous puissions les int\'egrer \`a la plateforme. Cela nous permettrait de cr\'eer des emplois et d'aider les personnes non habilit\'ees \`a b\'en\'eficier des services de la plateforme elles aussi.

    \paragraph{}
	Les cartes bancaires pouvant \^etre choisies comme l'une des options pour les solutions de paiement, Une \'evaluation devrait \^etre faite afin de permettre \`a l'entreprise de  s'adapter et se conformer \`a la norme \gls{pcidss} relative aux paiements \'electroniques via les cartes de cr\'edit.
